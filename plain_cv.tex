\documentclass[11pt,twoside,a4paper]{article}
\usepackage[margin=2cm]{geometry}
\usepackage{titling}
\usepackage{natbib}
\usepackage{mhchem, textcomp}
\usepackage{chemmacros}
\usepackage[font=small,labelfont=bf]{caption}
\usepackage{array}
\newcolumntype{L}[1]{>{\raggedright\let\newline\\\arraybackslash\hspace{0pt}}m{#1}}
\newcolumntype{C}[1]{>{\centering\let\newline\\\arraybackslash\hspace{0pt}}m{#1}}
\newcolumntype{R}[1]{>{\raggedleft\let\newline\\\arraybackslash\hspace{0pt}}m{#1}}

\bibliographystyle{model2-names}

\posttitle{\par\end{center}}
\predate{}
\postdate{}
\date{}
\preauthor{\begin{center}}
\postauthor{\par\end{center}\vspace{-0.5em}}
\setlength{\droptitle}{-60pt}

\title{Curriculum Vitae $\vert$ Robert Myhill}
\author{Wills Memorial Building, Queen’s Road, BRISTOL, BS8 1RJ, United Kingdom.\\
Tel: +44 (0) 117 33 15141, E-mail: bob.myhill@bristol.ac.uk\\
School of Earth Sciences, University of Bristol, \\
Date of Birth: 16th March 1986. Nationality: British.
}

\begin{document}

\maketitle

\subsection*{RELEVANT EXPERIENCE}
4 years PhD experience in earthquake location and waveform modelling including receiver function analysis, directivity measurements and cross correlation. 4 years postdoctoral experience in high pressure experimental petrology on melts, silicate and oxide phases, using piston cylinder and multi-anvil apparatus. Analytical experience includes EPMA, SEM, XRD, M\"{o}ssbauer, Raman and ERDA techniques. Experienced user of PerpleX and THERMOCALC thermodynamic software, and have implemented thermodynamic routines into the open-source BurnMan software package. I am fluent in the python and C++ programming languages.



\subsection*{EDUCATION}
\vspace{-0.5em}
\begin{table}[!h]
\centering
\begin{tabular}{L{2cm} L{10cm} L{5cm}}
2012 & PhD Geophysics \newline \emph{The Mechanisms of Deep Earthquakes} & University of Cambridge \\
2008 & MSci Natural Sciences (1/32 in class) \newline \emph{Earth Sciences} & University of Cambridge \\
2007 & MA Natural Sciences (1/36 in class) \newline \emph{Geology (plus Physics, Maths and Chemistry)} & University of Cambridge
\end{tabular}
\end{table}
\vspace{-1.5em}
\subsection*{PROFESSIONAL EXPERIENCE}

\vspace{-0.5em}
\begin{table}[!h]
\centering
\begin{tabular}{L{2cm} L{10cm} L{5cm}}
2016 & Postdoctoral Research Associate \newline \emph{Preparation for the InSight Mission} & University of Bristol \\
2015 & Postdoctoral Researcher \newline \emph{Thermodynamics of core formation} & Bayerisches Geoinstitut \\
2012-2014 & Alexander von Humboldt Research Fellow \newline \emph{High pressure experimental petrology} & Bayerisches Geoinstitut 
\end{tabular}
\end{table}
\vspace{-1.5em}

\subsection*{SELECTED HONOURS AND AWARDS}

\vspace{-0.5em}
\begin{table}[!h]
\centering
\begin{tabular}{L{2cm} L{13cm} L{2cm}}
2013-2014 & Alexander von Humboldt Research Fellowship for Postdoctoral Researchers. & \\
2011 & Outstanding Student Poster Award, Geodynamics Division (European Geophysical Union General Assembly). & \\
2010 & The Kingsley Bye-Fellowship. Magdalene College, Cambridge. & \\
2008 & The Hugo de Balsham Prize for Exceptional Academic Distinction. & \\
 & The Harkness Scholarship (first-placed Finalist in Geological Sciences, University of Cambridge). & \\
 & The Huppert Prize in Geophysics &  \\
2007 & The Henry Wilkinson Cookson Senior Scholarship in Natural Sciences. &
\end{tabular}
\end{table}

\vspace{-1.5em}

  
\subsection*{SELECTED COMMUNITY ROLES}
\begin{itemize}
\item Session convener at AGU Fall Meeting on seismology (2012), mineral physics (2015) and planetary sciences (2016)
\item Reviewer for the Journal of Mineralogical and Petrological Sciences, American Mineralogist, GeoResJ and Minerals
\item Chapter editor for Geochemical Perspectives
\item Developer of BurnMan and ASPECT software
\end{itemize}

\clearpage
\subsection*{PUBLICATIONS}
\subsubsection*{In preparation, submitted and accepted}
\small \sloppy
\begin{enumerate}
\item Myhill, R., Teanby, N. and Wookey, J., Seismic diagnostics for the interior chemistry of Mars: guides for the InSight Mission, in prep.
\item Myhill, R., Rubie, D. and Frost, D. J., Partitioning between silicate and metal melts; a model for core formation, in prep.
\item Dannberg, J. et al., Grain-size dependent convection and seismic observables in the Earth's mantle, in prep.
\item Myhill, R. et al., Quenchable water-rich aluminous post-stishovite, and implications for water cycling and seismic scatterers in the lower mantle, American Mineralogist, in prep.
\item Teanby, N. et al., Anelastic seismic coupling of wind noise through Mars’ regolith for NASA’s InSight Lander at short periods, submitted, Space Science Reviews.
\item Novella, D. et al., Melting phase relations in the systems Mg$_2$SiO$_4$-H$_2$O and MgSiO$_3$-H$_2$O at upper mantle conditions, in review, Geochimica et Cosmochimica Acta.
\item Myhill, R., Excess thermodynamic and elastic properties of mineral and melt solutions: modelling and implications for phase relations and seismic velocities, in revision, Contributions to Mineralogy and Petrology.
\end{enumerate}
\subsubsection*{Published}
\begin{enumerate}
\item Myhill, R. et al., 2016, Hydrous melting and partitioning in and above the mantle transition zone: insights from water-rich MgO-SiO$_2$-H$_2$O experiments, Geochimica et Cosmochimica Acta. doi:10.1016/j.gca.2016.05.027
\item Myhill, R. et al., 2016, On the P-T-\emph{f}O$_2$ stability of Fe$_4$O$_5$ and Fe$_5$O$_6$-rich phases: a thermodynamic and experimental study, Contributions to Mineralogy and Petrology, 171.5:1--11, doi:10.1007/s00410-016-1258-4.
 \item Frost, D. J. and Myhill, R., 2016, Chemistry of the Lower Mantle, in ``Deep Earth'' (AGU Geophysical Monograph), 225--240, doi:10.1002/9781118992487.ch18.
 \item Ishii, T. et al., 2016, Generation of pressures over 40 GPa using Kawai-type multi-anvil apparatus with tungsten carbide anvils, Review of Scientific Instruments, 87:024501, doi:10.1063/1.4941716.
 \item Rassios, A. et al., 2016, Preserving the non-preservable geoheritage of the Aliakmon River: A case study in geo-education leading to cutting-edge science, Bulletin of the Geological Society of Greece.
\item  Wessel, P. et al., 2015, Semiautomatic fracture zone tracking, Geochemistry, Geophysics, Geosystems, doi:10.1002/2015GC005853. 
\item Pamato, M. G., Myhill, R. et al., 2015, Lower mantle water reservoir implied by the extreme stability of a hydrous aluminosilicate, Nature Geoscience, 8:75--79, doi:10.1038/ngeo2306.
\item Myhill, R., 2013, Slab buckling and its effect on the distributions and focal mechanisms of deep-focus earthquakes, Geophysical Journal International, 192.2:837--853, doi:10.1093/gji/ggs054.
\item Myhill, R. and Warren, L. M., 2012, Fault plane orientations of deep earthquakes in the Izu-Bonin-Marianas subduction zone, Journal of Geophysical Research, 117:B06307, doi:10.1029/2011JB009047.
\item Myhill, R., McKenzie, D. and Priestley, K., 2011, The distribution of earthquake multiplets beneath the southwest Pacific, Earth and Planetary Science Letters, 301:87--97, doi:10.1016/j.epsl.2010.10.023.
\item Myhill, R., 2001, Constraints on evolution of the Mesohellenic Ophiolite from sub-ophiolitic metamorphic rocks, in Wakabayashi, J., and Dilek, Y., eds., M\'elanges: Processes of Formation and Societal Significance: Geological Society of America Special Paper 480:1--20, doi:10.1130/2011.2480(03).
\end{enumerate}
\end{document}


