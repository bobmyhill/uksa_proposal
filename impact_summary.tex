\documentclass[11pt,twoside,a4paper]{article}
\usepackage[margin=2cm]{geometry}
\usepackage{titling}
\usepackage{natbib}
\usepackage{mhchem, textcomp}
\usepackage{chemmacros}
\usepackage[font=small,labelfont=bf]{caption}
\usepackage{hyperref}
\hypersetup{hidelinks}

\bibliographystyle{model2-names}

\posttitle{\par\end{center}}
\predate{}
\postdate{}
\date{}
\preauthor{\begin{center}}
\postauthor{\par\end{center}\vspace{-0.5em}}
\setlength{\droptitle}{-60pt}

\title{The thermochemical evolution of Mars' deep interior: Geophysical insights with InSight}
\author{Robert Myhill (University of Bristol, UK)}

\begin{document}

%\maketitle
\section*{Impact Summary}


\subsection*{Public Outreach}

Space travel and planetary formation are both extremely popular topics, and the work in this project provides an ideal opportunity to engage the public. The work contained within this proposal will be used during Outreach events, such as those regularly conducted by @-Bristol. The PI has already conducted successful seismological outreach with @-Bristol. The School of Earth Sciences works regularly with this museum.

Bristol is also part of a recently funded UKSA public engagement project 'Marsquake Monitor: measuring the pulse of the planet', led by Paul Denton (British Geological Survey), which aims to provide school-age children with the opportunity to perform seismic experiments and analyse real mission data. The work from this project would provide excellent material to provide children with easily understandable and testable hypotheses.

\subsection*{Education}
The PI has recently been involved with the Global Summer School, hosted by Imperial College London, which aims to provide engineering-track A-level students with insights into real world applications. The school includes two focused days on Mars missions, and is already starting to incorporate InSight into the teaching schedule. The proposed work would provide a perfect scientific “big-picture” for teaching purposes.

The development of thermodynamic and thermoelastic models will be conducted with BurnMan, an easy-to-use, modular thermoelastic toolkit (Cottaar et al., 2013). One of the planned additions for BurnMan is a graphical planet builder, where a user can input a bulk composition and planetary structure and visualise the effects of structure and planetary evolution on observables such as mass, moment of inertia and seismic profiles. Two months of project time are devoted to implementing this functionality and an associated GUI, for use as an educational web-tool. One of the other developers of BurnMan (I. Rose) has developed a web applet to teach school children about mantle convection and global seismology (\url{http://ian-r-rose.github.io/interactive_earth/thermal_hires.html}), and the two projects could easily be combined to demonstrate the propagation of seismic waves on Mars.

\end{document}


